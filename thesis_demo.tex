\documentclass{newwuthesis}

% Use Times New Roman
\setmainfont{Times New Roman}

% Use math font that matches Times New Roman
\setmathfont{STIXTwoMath}[math-style=ISO]

% Re-calculate the lengths of a line of text with 45 or 65 characters using the current font
\setlxvchars[\normalfont\normalsize]
\setxlvchars[\normalfont\normalsize]
\checkandfixthelayout

% Show Layout
\usepackage{layout}

\begin{document}

\begin{equation}
    \label{eq:maxwell}
    \begin{aligned}
    \frac{\partial\mathcal{D}}{\partial t} & = \nabla\times\mathcal{H},   & \text{(Loi de Faraday)}\\
    \frac{\partial\mathcal{B}}{\partial t} & = -\nabla\times\mathcal{E},  & \text{(Loi d'Ampère)}\\
    \nabla\cdot\mathcal{B}                 & = 0,                         & \text{(Loi de Gauss)}\\
    \nabla\cdot\mathcal{D}                 & = 0.                         & \text{(Loi de Colomb)}
    \end{aligned}
\end{equation}

\[
    \oint_C {E \cdot d\ell = - \frac{d}{{dt}}} \int_S {B_n dA}
\]
\newpage

L45 = \the\xlvchars\par
L65 = \the\lxvchars

\newpage
\layout
\end{document}
